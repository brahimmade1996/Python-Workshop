\documentclass[12pt,a4paper]{article}


\usepackage[margin=1in]{geometry}
\usepackage[inline]{enumitem}
\makeatletter
\newcommand{\inlineitem}[1][]{%
\ifnum\enit@type=\tw@
			{\descriptionlabel{#1}}
		\hspace{\labelsep}%
\else
	\ifnum\enit@type=\z@
			\refstepcounter{\@listctr}\fi
		\quad\@itemlabel\hspace{\labelsep}%
\fi}
\makeatother
\parindent=0pt

\newtheorem{theorem}{قضیه}[section]
\newtheorem{definition}{تعریف}

\usepackage{xepersian}
\settextfont[Scale=1]{B Nazanin+ Regular}
\setdigitfont[Scale=1]{B Nazanin+ Regular}
\setlatintextfont{Calibri}
\baselineskip=0.5cm

\newcommand{\ن}{‌}

\title{}
\author{}
\date{}

\begin{document}
	\maketitle{}
	\section{تمارین سری یک}
	\begin{enumerate}
	\item
برنامه‌ای بنویسید که ۳ ورودی از کاربر بگیرد و آنها را بر حسب بزرگتری مرتب کند.
	\item
	تابعی بنویسید که عددی صحیحی بگیرد و فاکتوریل آن را حساب کند.
	\item 
	برنامه‌ای بنویسید که یک عدد باینری بگیرد و آن را به صورت ده دهی نشان دهد.
	\item
	برنامه‌ای بنویسید که نام و نام‌خانوادگی و سال تولد کاربر را بگیرد و به او بگویید که چند سال اش است.\\ (نام + نام‌خاموادگی + <سن> + سال سن دارد)
	\item 
	تابعی بنویسید که nامین عدد اول را برگرداند.
	\item
	برنامه‌ای بنویسید که جمع فاکتوریل تمام عداد کمتر از n را برگرداند.
	\item 
	برنامه‌ای بنویسید که جمع لگاریتم nتا عدد اول را برمیگرداند. 
	\end{enumerate}
	\section{تمارین سری دوم}
	\begin{enumerate}
	\item 
	برنامه‌ای بنویسید که تعداد رشته‌های bob داخل یک رشته را بدست بیاورد.
	\item
	توابعی به نویسید که بتوان با آن برداشن و واریزها با توجه به مالیات( ۰.۰۵ درصد) افزوده بر آن حساب کند.
	\item
	 اگر در تمرین قبل اگر در یک سال حسابی با ۵۰۰۰ پول داشته باشیم و سود ماهیانه 1.5 درصد باشد و ماهی ۲ درصد در از حساب برداشت کرده باشیم و قبل از اتمام ماه آخر ۱۰۰۰ هم به حسابمان وارد شده باشد، آخر سال چه قدر پول در حسابمان باقی می ماند؟
	 \item 
	 تابعی بنویسید که لیستی دو بعدی را بگیرد و متقارن یا نامتقارن بودن آن را مشخص کند

	\item
	لیستی از اسامی دانشگاه‌ها، تعداد پذیرفته شدگان و هزینه زندگی یک دانشجو وجود دارد،
	برنامه‌ای بنویسید که تعداد دانشجویان هر دانشگاه و هزینه آنها را به علاوه تعداد کل و هزینه کل نمایش دهد.\\
	\begin{latin}
	\lr{usa\_univs =}\\
 \lr{[ [`California Institute of Technology',2175,37704],}\\
 \lr{[`Harvard',19627,39849],}\\
 \lr{[`Massachusetts Institute of Technology',10566,40732],}\\
 \lr{[`Princeton',7802,37000],}\\
 \lr{[`Rice',5879,35551],}\\
 \lr{[`Stanford',19535,40569],}\\
 \lr{[`Yale',11701,40500]]}
 \end{latin}
 \item 
 برنامه‌ای بنویسید که یک لیست از اعداد را بگیرد و بزرگترین، میانگین و ضرب آنها را بنویسید.
 \item 
 برنامه‌ای بنویسید که یک عدد اعشاری بگیرد و تعداد ارقام صحیح آن را بنویسد.
 \item
 برنامه‌ای بنویسید که تعداد حروف صدا دار یک رشته را بگوید.
	\end{enumerate}
\section{تمارین سری سوم}
\begin{enumerate}
\item
برنامه‌ای برای بازی ۲۰ سوالی بنویسید.
\begin{latin}
\lr{words=}\\
\lr{[`after',`afternoon',`ball',`book',`clock',`computer',`door',`duck',`first',`girl',`glass',`hello',}\\
\lr{`human',`mathematics',`minute`,`monkey',`morning',`pen',`physics',`quick',`school',`second',}\\
\lr{`student',`table',``teacher'', `tree',`university',`woman',`world',`zoo']}
\end{latin}
\item
تابعی بنویسید که یک چندجمله‌ای بگیرد و مشتق آن را برگرداند.
\end{enumerate}
\section{تمارین سری چهارم}
\begin{enumerate}

\item
برنامه‌ای بنویسید که بزرگترین زیررشته از یک رشته که برحسب حروف‌الفبا است را به ما برگرداند.
\item
بازی ۲۰ با خواندن تکستی از لغات بنویسید.
\item 
برنامه‌ای بنویسید که کاربر به شما زمانی بده و شما آن را به حسب ثانیه برگردانید.
\item
کلاسی طرحی کنید که سیستم رمز سزار و سیستم پیشرفته رمز سزار را در آن اجرا شود.

\end{enumerate}

\section{تمارین سری پنجم}
\begin{enumerate}
\item
برنامه‌ای بنویسید که یک ماتریس بگیرد و به روش اشلون وارون آن را پیدا کند.
\item

\end{enumerate}


	

\end{document}
